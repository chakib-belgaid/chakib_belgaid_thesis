\clearpage
\chapter{Discussion and Conclusion}
\label{chapter:conclusion}

This manuscript reports on several contributions to measuring and reducing software energy consumption.
We used a three-step strategy to lower software energy: benchmarking, measuring, and optimizing.
We started with the benchmarking phase.
Chapter~\ref{chapter:literature_review} discussed the challenges of a successful benchmarking strategy: reproducibility, accuracy, and representativeness.
We concluded that software containers, like "Docker", would be the best fit to ensure that energy studies could be reproduced.
We then extended this reproducibility to an evolving protocol that helps researchers keep up with the rapid pace of software development.
Then, we targeted the accuracy by studying the hardware and software factors that can impact the energy variation and how practitioners can tune them to harness this energy variation.

After establishing a robust benchmarking protocol to create energy-based experiments, we shifted our focus to the optimization side.
We opted to start with Python, the most popular yet energy-hungry programming language.
As a result, we began by examining the energy behavior of Python code in its most common usage. 
Then, we presented a non-intrusive method to lessen its energy use.
Following that, we applied the same strategy to another programming language known for its legacy code base, Java, to prove that we can still cut the energy usage of existing running applications without incurring high costs.

Lastly, we used the flexibility of the micro-services architecture to look at how each programming language uses energy in different web scenarios.
We first examined the effects of the various programming languages when dealing with the \emph{Remote Procedure Call} (RPC) protocol.
Then, we extended this study to a more practical application by comparing $261$ web frameworks, each implementing the same website using seven use cases.
Then, we provided practitioners with a dashboard to determine which stack is best for a given situation.


\section{Summary of Contributions}
\label{section:SummaryofContributions}
The contributions reported in this thesis are covered in this section.
The following is a summary of them:

\subsection{Published Articles}
This part summarizes the contributions that have been already accepted in conferences and journals.

\begin{enumerate}
      \item \textbf{Taming energy consumption variations in systems benchmarking}:
      We investigate the phenomenon of variation when measuring the energy consumption of experiments in this study.
      In this paper, we discuss various hardware and software factors that can amplify variations in recorded energy measures, with a focus on the following research questions:
      \begin{description}
            \item[\textsc{RQ}~1:] Does the benchmarking protocol affect the energy variation?
            \item[\textsc{RQ}~2:] How important is the impact of the processor features on the energy variation?
            \item[\textsc{RQ}~3:] What effect does the operating system have on energy variation?
            \item[\textsc{RQ}~4:] Does the choice of processor make a difference in reducing the energy variation?
      \end{description}

      This contribution shows how processor features can significantly affect the variation in energy use between the benchmarking protocol and the operating system.
      Finally, this study presents several guidelines for controllable parameters that practitioners could easily change to increase the accuracy of their experiments.

      \bibentry{ournani2020taming}


      \item \textbf{Evaluating the impact of Java virtual machines on energy consumption}:
      In this paper, we thoroughly investigate how JVMs affect software energy usage.
      To address the following research concerns, we reveal through this study several trials on hundreds of JVMs versions provided by various providers:
      \begin{description}
            \item[\textsc{RQ}~1:] What is the impact of existing JVM distributions on the energy consumption of Java-based software services?
            \item [\textsc{RQ}~1:] What are the relevant JVM settings that can reduce the energy consumption of a given software service?
      \end{description}
      The findings demonstrate that choosing the right JVM platform can significantly reduce energy usage depending on the software and use case.
      This optimization can also be achieved by properly configuring the JIT and GC parameters.

      \bibentry{ournani2021evaluating}
\end{enumerate}

\subsection{Softwares and Tools}
While the benchmarking and optimization parts were presented as chapters in this thesis, the measurement part was primarily based on developing tools that measure the energy consumption for a given use case.
The following is a summary of the tools that were developed during this thesis:
\begin{itemize}
      \item \textbf{Jouleit} (\url{github.com/powerapi-ng/jouleit}): a tool that can be used to monitor energy consumption for any Linux program, this tool was used to compare the energy consumption of different JVMs;
      \item \textbf{JRefferal} (\url{github.com/chakib-belgaid/jreferral}): a tool that allows the user to explore the JVM settings and their impact on the energy consumption of a given Java program. This tool was the result of the second article of this thesis;
      \item \textbf{PyJoules} (\url{pypi.org/project/pyJoules}) is a software toolkit to measure the energy footprint of a host machine along the execution of a piece of Python code. It can measure the energy consumption on the level of script, function, and bloc of code;
      \item \textbf{JouleHunter} (\url{pypi.org/project/joulehunter}): an energy profiler for python applications. It can be used to highlight the functions that consume the most energy in a given Python program. Its main usage is to help developers do an exploratory analysis of their application to scope the functions that should be optimized to be then targeted by PyJoules;
      \item \textbf{GreenBoard} (\url{github.com/chakib-belgaid/greenboard}): a dashboard designed to help developers choose the best stack for their web application. It is based on the results of the third article of the last chapter.
\end{itemize}



\subsection{Future Submissions}
\begin{itemize}
      \item Reducing the energy consumption of Python using non-intrusive techniques,
      \item Empirical analysis on the energy consumption of different web frameworks,
      \item The impact of programming languages on energy consumption of web services (a case study of RPC protocol),
      \item THow do ORMs affect how much energy is used? A case study of Django and Flask.
\end{itemize}

\section{Future Work\note{missing}}
.... Some potential areas for future efforts could include the following:

\begin{enumerate}
      \item ...
      \item ...
      \item ...

\end{enumerate}

\vfill \strut  % to fill the rest of the page with blank lines
\cleardoublepage