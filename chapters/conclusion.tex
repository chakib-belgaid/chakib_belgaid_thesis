\clearpage
\chapter{Discussion and Conclusion}
\label{chapter:conclusion}

This manuscript reports on several contributions to measuring and reducing software energy consumption.
We used a three-step strategy to lower software energy: benchmarking, measuring, and optimizing.
We started with the benchmarking phase.
Chapter~\ref{chapter:literature_review} discussed the challenges of a successful benchmarking strategy: reproducibility, accuracy, and representativeness.
We concluded that software containers, like "Docker", would be the best fit to ensure that energy studies could be reproduced.
We then extended this reproducibility to an evolving protocol that helps researchers keep up with the rapid pace of software development.
Then, we targeted the accuracy by studying the hardware and software factors that can impact the energy variation and how practitioners can tune them to harness this energy variation.

After establishing a robust benchmarking protocol to create energy-based experiments, we shifted our focus to the optimization side.
We opted to start with Python, the most popular yet energy-hungry programming language.
As a result, we began by examining the energy behavior of Python code in its most common usage.
Then, we presented a non-intrusive method to lessen its energy use.
Following that, we applied the same strategy to another programming language known for its legacy code base, Java, to prove that we can still cut the energy usage of existing running applications without incurring high costs.

Lastly, we used the flexibility of the micro-services architecture to look at how each programming language uses energy in different web scenarios.
We first examined the effects of the various programming languages when dealing with the \emph{Remote Procedure Call} (RPC) protocol.
Then, we extended this study to a more practical application by comparing $261$ web frameworks, each implementing the same website using seven use cases.
Then, we provided practitioners with a dashboard to determine which stack is best for a given situation.


\section{Summary of Contributions}
\label{section:SummaryofContributions}
The contributions reported in this thesis are covered in this section.
The following is a summary of them:

\subsection{Published Papers}
This part summarizes the contributions that have been already accepted in conferences and journals.

\begin{enumerate}
      \item \textbf{Taming energy consumption variations in systems benchmarking}:
            We investigate the phenomenon of variation when measuring the energy consumption of experiments in this study.
            In this paper, we discuss various hardware and software factors that can amplify variations in recorded energy measures, with a focus on the following research questions:
            \begin{description}
                  \item[\textsc{RQ}~1:] Does the benchmarking protocol affect the energy variation?
                  \item[\textsc{RQ}~2:] How important is the impact of the processor features on the energy variation?
                  \item[\textsc{RQ}~3:] What effect does the operating system have on energy variation?
                  \item[\textsc{RQ}~4:] Does the choice of processor make a difference in reducing the energy variation?
            \end{description}

            This contribution shows how processor features can significantly affect the variation in energy use between the benchmarking protocol and the operating system.
            Finally, this study presents several guidelines for controllable parameters that practitioners could easily change to increase the accuracy of their experiments.

            \bibentry{ournani2020taming}


      \item \textbf{Evaluating the impact of Java virtual machines on energy consumption}:
            In this paper, we thoroughly investigate how JVMs affect software energy usage.
            To address the following research concerns, we reveal through this study several trials on hundreds of JVMs versions provided by various providers:
            \begin{description}
                  \item[\textsc{RQ}~1:] What is the impact of existing JVM distributions on the energy consumption of Java-based software services?
                  \item [\textsc{RQ}~1:] What are the relevant JVM settings that can reduce the energy consumption of a given software service?
            \end{description}
            The findings demonstrate that choosing the right JVM platform can significantly reduce energy usage depending on the software and use case.
            This optimization can also be achieved by properly configuring the JIT and GC parameters.

            \bibentry{ournani2021evaluating}
\end{enumerate}

\subsection{Softwares and Tools}
While the benchmarking and optimization parts were presented as chapters in this thesis, the measurement part was primarily based on developing tools that measure the energy consumption for a given use case.
The following is a summary of the tools that were developed during this thesis:
\begin{itemize}
      \item \textbf{Jouleit} (\url{github.com/powerapi-ng/jouleit}): a tool that can be used to monitor energy consumption for any Linux program, this tool was used to compare the energy consumption of different JVMs;
      \item \textbf{JRefferal} (\url{github.com/chakib-belgaid/jreferral}): a tool that allows the user to explore the JVM settings and their impact on the energy consumption of a given Java program. This tool was the result of the second article of this thesis;
      \item \textbf{PyJoules} (\url{pypi.org/project/pyJoules}) is a software toolkit to measure the energy footprint of a host machine along the execution of a piece of Python code. It can measure the energy consumption on the level of script, function, and bloc of code;
      \item \textbf{JouleHunter} (\url{pypi.org/project/joulehunter}): an energy profiler for python applications. It can be used to highlight the functions that consume the most energy in a given Python program. Its main usage is to help developers do an exploratory analysis of their application to scope the functions that should be optimized to be then targeted by PyJoules;
      \item \textbf{GreenBoard} (\url{github.com/chakib-belgaid/greenboard}): a dashboard designed to help developers choose the best stack for their web application. It is based on the results of the third article of the last chapter.
\end{itemize}



\subsection{Future Submissions}
\begin{itemize}
      \item Reducing the energy consumption of Python using non-intrusive techniques,
      \item Empirical analysis on the energy consumption of different web frameworks,
      \item The impact of programming languages on energy consumption of web services (a case study of RPC protocol),
      \item How do ORMs affect how much energy is used? A case study of Django .
\end{itemize}

\section{Future Work\note{ In Draft }}
While our contributions are a good start to the energy-aware software engineering field, there are still many challenges to overcome. This is just the tip of the iceberg. The following are some of the challenges that we would like to address in the future:

\subsection{Short Term Challenges}
Before starting a journey looking for a new mine, one should first look at the resources available to him. In the same way, we need to first look at the resources available to us before we start looking for new challenges. The following are some of the challenges that we can address in the short term:

\paragraph{The evolution of python interpreters}
In Chapter \ref{chapter:python} we compared several python interpeters. However, most of these alternative solutions were based on python2 which is now deprecated. On the other hand, the default python interpreter (aka . CPython) has included many features and optimization since our last study, such as the introduction of Python introduced the \texttt{dataclasses} in version 3.7 (PEP 557\footnote{\url{https://peps.python.org/pep-0557/}}) that can be used to reduce the memory footprint of python objects,
the new parser in python 3.9~\footnote{\url{https://docs.python.org/3/whatsnew/3.9.html}}, the user type alias in python3.10~\footnote{\url{https://peps.python.org/pep-0613/}}.
And the most interesting changes for us occurred in CPython 3.11 which is claimed to be 25\% faster than python 3.10 \footnote{\url{https://github.com/faster-cpython/ideas}}. We need to re-evaluate the impact of these changes on the energy consumption of python programs.

\paragraph{The impact of ORMS}
We continue with our work with Python, and this time we will delve deeper into the impact of the ORMS, As shown in Section\ref{sec:webdev}, The ORMs are the most energy-consuming part of a web application. We intend to widen this analysis to other python ORMS such as SQLAlchemy\link{https://www.sqlalchemy.org/} and Peewee\link{http://docs.peewee-orm.com/en/latest/}. We will also explore the real relationship between the ORM, the database, and the web framework.

\paragraph{Python and machine learning}
In chapter \ref{chapter:python}, we showed that training accuracy has a huge cost on the energy using a single model. However, in the real world, we often use multiple models to solve a problem. We will explore the impact of using multiple models on the energy consumption of a machine learning application As well as compare the energy consumption of different machine learning frameworks such as PyTorch\link{https://pytorch.org/}, scikit-learn\link{https://scikit-learn.org/stable/} and TensorFlow \link{https://www.tensorflow.org/}.



\paragraph{ Which is the greenest couple? a comparative study of JVMs and JVM-based programming languages}
As we have seen in Chapter \ref{chapter:java} the choice of the JVM can greatly impact the energy consumption of the program. On the other hand, we have seen in Chapter \ref{chapter:porgramming_langauges} that other JVM-based languages depicted half power consumption of the JAVA code, was it only because of the web framework ? or there is a real difference between the JVMs and the JVM based languages? We will explore this question by comparing the energy consumption of different JVMs and JVM-based languages such as Scala, Kotlin, Groovy, and Clojure.

\paragraph{the impact of programming languages on energy consumption, do translators helps in reducing the energy cost of a program?}
One of the hardest challenges when comparing the energy consumption of multiple programming languages was the bias of the expertise of the programmer. One solution was to create a reference benchmark that allows one to compare several programming lanagues \link{https://benchmarksgame-team.pages.debian.net/benchmarksgame/index.html}, this was used for many researchers such as the work of \citeauthor{couto2017towards} \cite{couto2017towards}, others were to use some basic algorithms like \cite{noureddine_preliminary_2012} where they compare the energy consumption using the Hanoi tower problem \link{https://en.wikipedia.org/wiki/Tower_of_Hanoi} using different programming languages using the same algorithm.or other simple benchmarks using the Rosettacode base \link{https://rosettacode.org/wiki/Rosetta_Code}
however this scope is limited to single algorithms. and does not help cover the production mode, this is why we shifted to the web frameworks.  Now that are semantinc analysers issued from the openai project\link{https://openai.com/} like the automatic test generator ponicode\link{https://www.ponicode.com/} and the ai code geneator \link{https://github.com/features/copilot}
an interesting new feature provided by the GitHub copilot is github copilot labs\link{https://githubnext.com/projects/copilot-labs/} is the ability to automatically translate the code from one language to another. We will explore the impact of this feature on the energy consumption of a program.


\paragraph{raising awareness of the energy consumption of the Softwares}
While this thesis's main focus was to optimize energy consumption using comparative studies, it was easier to say that approach X is greener than approach Y no matter which metric we were using. however, it won't be the case for developers while measuring the energy consumption of their programs, some tried to give labels such as (A,B,C ..etc) while others transtaled this raw metrics in an equivalent of a fuel consumption (e.g. 1 liter of fuel per 1000 lines of code). most of the approaches where to use the carbon emmision, such as\cite{patterson2021carbon} or ecograder~\link{https://ecograder.com/}
we will explore these approaches and see how they can be used to rise the awareness of the energy consumption of their code





\begin{enumerate}

      \item Deeper analysis for some web frameworks ( what is the reason behind an increase in power consumption while scaling )



\end{enumerate}


\subsection*{Representativeness}:
What is the purpose of doing optimizations if they cannot be applied to real-world applications?
\begin{enumerate}
      \item CI/CD energy tests  - joulediff -
      \item Green commits
      \item lazy code, sacrificing the performance in order to get a better energy consumption --> green faas
      \item more representative benchmarks, using several CPU saturation levels
      \item measuring the energy impact instead of the row energy consumption. ( the highway analogy when a single application consumes less energy, however the total energy consumption of the system increases )
      \item https://www.oneapi.io/open-source/
\end{enumerate}
\vfill \strut  % to fill the rest of the page with blank lines
\cleardoublepage