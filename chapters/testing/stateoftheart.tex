%  State of the art. 
%  TODO : 

% - add : Evaluate Collaboratory article 
% - add : the  hardware papers 

\paragraph{Studying Hardware Factors.}

This variation has often been related to the manufacturing process~\cite{coles_comparing_2014}, but has also been a subject of many studies, considering several aspects that could impact and vary the energy consumption across executions and on different chips.
On the one hand, the correlation between the processor temperature and the energy consumption was one of the most explored paths.
Kistowski~\emph{et~al.} showed in~\cite{joakim_v_kisroski_variations_2016} that identical processors can exhibit significant energy consumption variation with no close correlation with the processor temperature and performance.
On the other hand, the authors of~\cite{wang_potential_2018} claimed that the processor thermal effect is one of the most contributing factors to the energy variation, and the correlation between the CPU temperature and the energy consumption variation is very tight.

\note{add the corelation value}

This makes the processor temperature a delicate factor to consider while comparing energy consumption variations across a set of homogeneous processors.%, but also, if a same workload can cause the CPUs to reach different temperatures.

The ambient temperature was also discussed in many papers as a candidate factor for the energy variation of a processor.
In~\cite{ranka_energy_2009}, the authors claimed that energy consumption may vary due to fluctuations caused by the external environment.
These fluctuations may alter the processor temperature and its energy consumption.
However, the temperature inside a data center does not show major variations from one node to another.
In~\cite{el_mehdi_diouri_your_2013}, El~Mehdi~Dirouri~\emph{et~al.} showed that switching the spot of two servers does not affect their energy consumption.
Moreover, changing hardware components, such as the hard drive, the memory or even the power supply, does not affect the energy variation of a node, making it mainly related to the processor.
This result was recently assessed by~\cite{wang_potential_2018}, where the rack placement and power supply introduced a maximum of $2.8\,\%$ variation in the observed energy consumption.

Beyond hardware components, the accuracy of power meters has also been questioned.
Inadomi~\emph{et~al.}~\cite{inadomi_analyzing_2015} used three different power measurement tools: RAPL, Power Insight\footnote{\url{https://www.itssolution.com/products/trellis-power-insight-application}} and BGQ EMON.
All of the three tools recorded the same $10\,\%$ of energy variation, that was supposedly related to the manufacturing process.
%% this is to prove that the problem is REAL !!! 
\paragraph{Mitigating Energy Variations.}
Acknowledging the energy variation problem on processors, some papers proposed contributions to reduce and mitigate this variation.
In~\cite{inadomi_analyzing_2015}, the authors introduced a variation-aware algorithm that improves application performance under a power constraint by determining module-level (individual processor and associated DRAM) power allocation, with up to $5.4\times$ speedup.
The authors of~\cite{hammouda_noise-tolerant_2015} proposed parallel algorithms that tolerate the variability and the non-uniformity by decoupling per process communication over the available CPU.
Acun~\emph{et~al.}~\cite{acun_variation_2016} found out a way to reduce the energy variation on Ivy~Bridge and Sandy~Bridge processors, by disabling the Turbo~Boost feature to stabilize the execution time over a set of processors.
They also proposed some guidelines to reduce this variation by replacing the old slow chips, by load balancing the workload on the CPU cores and leaving one core idle.
They claimed that the variation between the processor cores is insignificant.
In~\cite{chasapis_runtime-guided_2016}, the researchers showed how a parallel system can be used to deal with the energy variation by compensating the uneven effects of power capping.

In~\cite{marathe_empirical_2017_m}, the authors highlight the increase of energy variation across the latest Intel micro-architectures by a factor of $4$ from Sandy~Bridge to Broadwell, a $15\,\%$ of run-to-run variation within the same processor and the increase of the inter-cores variation from $2.5\,\%$ to $5\,\%$ due to hardware-enforced constraints, concluding with some recommendations for Broadwell usage, such as running one hyper-thread per core.
%  NOTE :/this might be interesting for greeFaaS


\subsection*{Our position}
Meanwhile the previoius work was to create a large umbrella tht englobes all emprical tests related to computer science, we want to narrow our study to energy testing only. In other words we will study pitfalls that hinders the energy claimes of software .

We are fully aware of the impact that hardware has on energy variation. However, we believe that there is still a room to reduce this energy variation for practicioners using only paramenters that they are in cotrol.
To do so, we have inducted a set of empirical experiments using the guidelines provided by state of the art, to determine which controllable factors can reduce the variation of the energy consumption of tests.
We will first start with evaluating the paramenters mentioned by \cite{heinrich_predicting}.

