\chapter{Energy Footprint of Programming Languages}
\label{chapter:porgramming_langauges}
\section{Introduction}
In this chapter, we study how the programming language affects the energy the software consumes.
We suggest starting with general micro-benchmarking and watching how each programming language performs with the CPU and memory.
The main goal of this chapter is to advise developers on how to choose a programming language based on their project's needs in order to make their product use the least amount of energy possible.
For such a question, no answer is obvious.
Nonetheless, there are some features we can take from each programming language, such as:
\begin{itemize}
    \item performance,
    \item community support,
    \item scalability,
    \item energy consumption,
    \item memory usage.
\end{itemize}

As we saw in the last chapter, one of the most important things about a test is how well it \textsc{represents} the production environment.
Therefore, we extend this study to include real-world use cases, with two case studies provided in the parts that follow.

\section{Investigating Remote Procedure Call Frameworks}
\import{\currfiledir/rpc}{rpc}
\clearpage

\section{Investigating Web Application Frameworks}
\import{\currfiledir/web_frameworks}{web_frameworks}

\section{Conclusion\note{missing}}
