\chapter{Energy footprint of programming languages}
\label{chapter:porgramming_langauges}
In this chapter, we discuss the impact of the choice of the programming language on the energy consumption of the resulting software.
To do so, we suggest to start with the general micro benchmarking and see how each programming languages interact with the CPU/Memory.

The ultimate goal of this chapter is to provide a guideline on which programming language the developers should chose based on the charecteristics of the project in order to minimize the energy foot print of their product.
No answer is evidedent for a such question. However, we can extract some feartures of each programming langues such as
\begin{itemize}
    \item performance
    \item community support
    \item scalability
    \item energy consumptiom
    \item memory usage
    \item etc
\end{itemize}

first we will start but analysing the behaviour of the general purpose programming langauges with some micro benchkaks, principally for the CLBG game and others from rosetta code base


As we have seen in the previous chapter, one of the most important feature of a test is to be \textbf{representative}.
Therefore, we extend this study to some real-life use case, and the following sections below provide two case studies.
\section{Remote Procedure Call Frameworks}
\import{\currfiledir/rpc}{rpc}
\section{web Frameworks}
\import{\currfiledir/web_frameworks}{web_frameworks}