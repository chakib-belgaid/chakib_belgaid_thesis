\chapter{Energy footprint of programming languages}
\label{chapter:porgramming_langauges}
In this chapter, we talk about how the programming language affects how much energy the software consumes.We suggest starting with general microbenchmarking and watching how each programming language performs with the CPU and memory.The main goal of this chapter is to advise developers on how to choose a programming language based on their project's needs in order to make their product use the least amount of energy possible. For such a question, no answer is evidedent.But there are some features we can take from each programming language, such as: 
\begin{itemize}
    \item performance
    \item community support
    \item scalability
    \item energy consumptiom
    \item memory usage
\end{itemize}

As we saw in the last chapter, one of the most important things about a test is how well it \textsc{represents} the production environment. Therefore, we extend this study to include real-world use cases, with two case studies provided in the parts that follow. 
\section{Remote Procedure Call Frameworks}
\import{\currfiledir/rpc}{rpc}
\section{web Frameworks}
\import{\currfiledir/web_frameworks}{web_frameworks}