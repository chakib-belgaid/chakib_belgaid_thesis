\chapter{Energy Footprint of Distributed Programming Frameworks}
\label{chapter:porgramming_langauges}

% TODO : repphrase this 

After looking at Django's studies, we wanted to expand our study to other programming languages and frameworks.
The idea behind this study is to expand the work of Pereira et al. from the micro benchmarking level to get closer to real-world applications.
First, we will start with studying the RPC protocol since it offers a multi-language interface. Therefore we will reduce the developer expertise bias
after that; we will enlarge this study to cover the most popular web frameworks. In this study, we will focus on the energy footprint of the web frameworks and the RPC protocol in the context of an unlimited time. Therefore the factor time will have less impact, unlike the micro benchmarking study.

This chapter studies how the programming framework affects the energy the software consumes.
We suggest starting with general micro-benchmarking and watching how each programming framework performs with the CPU and memory.
The main goal of this chapter is to advise developers on how to choose a programming framework based on their project's needs to make their product use the least amount of energy possible.
For such a question, no answer is obvious.
Nonetheless, there are some features we can take from each programming framework, such as:
\begin{itemize}
    \item performance,
    \item community support,
    \item scalability,
    \item energy consumption,
    \item memory usage.
\end{itemize}

As we saw in the previous chapter, one of the most important things about a benchmark is how well it \textsc{reflects} the production environment.
Therefore, we extend our study to cover real-world use cases, including two case studies reported in the following sections.

\section{Investigating Remote Procedure Call Frameworks}
\import{\currfiledir/rpc}{rpc}


\section{Investigating Web Application Frameworks}
\import{\currfiledir/web_frameworks}{web_frameworks}

\section{Conclusion \note{missing}}
