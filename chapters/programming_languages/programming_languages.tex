\chapter{Energy Footprint of Distributed Programming Frameworks}
\label{chapter:porgramming_langauges}
% \section{Introduction}
In this chapter, we study how the programming framework affects the energy the software consumes.
We suggest starting with general micro-benchmarking and watching how each programming framework performs with the CPU and memory.
The main goal of this chapter is to advise developers on how to choose a programming framework based on their project's needs to make their product use the least amount of energy possible.
For such a question, no answer is obvious.
Nonetheless, there are some features we can take from each programming framework, such as:
\begin{itemize}
    \item performance,
    \item community support,
    \item scalability,
    \item energy consumption,
    \item memory usage.
\end{itemize}

As we saw in the previous chapter, one of the most important things about a benchmark is how well it \textsc{reflects} the production environment.
Therefore, we extend our study to cover real-world use cases, including two case studies reported in the following sections.

\section{Investigating Remote Procedure Call Frameworks}
\import{\currfiledir/rpc}{rpc}


\section{Investigating Web Application Frameworks}
\import{\currfiledir/web_frameworks}{web_frameworks}

\section{Conclusion \note{missing}}
