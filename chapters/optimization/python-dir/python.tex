\section{Introduction}
Dynamic programming languages, except Perl, have surpassed compiled programming languages in terms of popularity among software system developers over the past decade (cf. \Cref{fig:pypl}).
However, it remains unclear whether this category of dynamic programming languages can truly compete with compiled ones in terms of power consumption.

\begin{figure}[htbp]
    \includegraphics[width=\linewidth]{imgs/programminglanguangespopularity.png}
    \caption{PYPL Popularity of Programming Languages~\cite{noauthor_pypl_nodate}.}
    \label{fig:pypl}
\end{figure}

In particular, Noureddine~\emph{et~al.}~\cite{noureddine_preliminary_2012} in 2012, and then Pereira~\emph{et~al.}~\cite{pereira_energy_2017} in 2017, conducted empirical power measurements on this topic, and both concluded that compiled programming languages overcome dynamic ones when it comes to power consumption.
According to their experiments, an interpreted programming language, like Python, can impose up to a $7,588\,\%$ energy overhead compared to C~\cite{pereira_energy_2017} (cf. \Cref{fig:hannoi}).
In this chapter, we explore the oblivious optimizations that can be applied to Python legacy applications to reduce their energy footprint.
As Python is widely adopted by software services deployed in public and private cloud infrastructures, we believe that our contributions will benefit a wide diversity of legacy systems and not only favorably contribute to reducing the carbon emissions of ICT, but also reduce their cloud invoice for the resources consumed by these services.
More specifically, this chapter focuses on runtime optimizations that can be adopted by developers to leverage the power consumption of Python applications.
We start by studying the impact of programmer choices, such as the type of data or control structures, on the global energy consumption of the execution code.
Then, we discuss other factors, such as the levels of concurrency, before we investigate other non-intrusive approaches to optimize the energy consumption of applications.
One type of optimization includes alternative interpreters and libraries that are dedicated to optimizing the code without changing its structures, such as \emph{ahead-of-time} (AOT) compilation and \emph{just-in-time} (JIT) libraries that are maintained by the community.

\begin{figure}[htbp]
    \includegraphics[width=\linewidth]{imgs/hannoiimplementation.png}
    \caption{Energy consumption of a recursive implementation of Tower of Hanoi program in different languages~\cite{noureddine_preliminary_2012}}
    \label{fig:hannoi}
\end{figure}

\section{Motivation}

\subsection{Python Popularity}
Nowadays, Python seems to attract a large community of developers who are interested in data analysis, web development, system administration, and machine learning.
According to a survey conducted in 2018 by JetBrains,\footnote{\url{https://www.jetbrains.com/research/python-developers-survey-2018/}} one can fear that the wide adoption of dynamic programming languages, like Python, in production may critically hamper the power consumption of ICT.
As the popularity of such dynamic programming languages partly builds on the wealth and the diversity of their ecosystem (\emph{e.g.}, the NumPY, SciKit\,Learn, and Panda libraries in Python), one cannot reasonably expect that developers will likely move to an alternative programming language mostly for energy considerations.
Rather, we believe that a better option consists of leveraging the strength of this rich ecosystem to promote energy-efficient solutions to improve the power consumption of legacy software systems.

\subsection{Python Gluttony}
% In the other hand,because of the interpreted nature of Python code, this programming language has paid a high price in terms of performance.  \ref{fig:clbg}, memory and energy consumption.
According to~\cite{pinto_energy_2017} and~\cite{noureddine_preliminary_2012}, Python tends to be one more energy hungry programming language.
As one can notice in \Cref{fig:hannoi}, Python consumes $30$ times more than C or C++.
The benchmark was done with an implementation of the \fnurl{Tower of Hanoi}{ https://en.wikipedia.org/wiki/Tower_of_Hanoi} of 30 disks.

Python consumes a lot of energy, mainly because it is slow in execution.
Its flexibility and simplicity caused it to drop off in performance because Python gains its flexibility from being a dynamic language.
Therefore, it needs an interpreter to execute its programs, which makes them much slower compared to the others that are written in compiled programming languages, such as C and C++ or semi-compiled languages like Java.

As shown in \Cref{fig:clbg}, one can observe that, for most of the applications taken for the \emph{Computer Language Benchmark Game} (CLBG), Python takes more time to execute---the only case that he was not the worst one was in the benchmark \textsf{regx-redux} where he beats Go---and in some cases the gap was huge, such as in \textsf{n-body} where Python took around $100$ times more than C++.\footnote{\url{https://benchmarksgame-team.pages.debian.net/benchmarksgame/index.html}}

\begin{table}[hbt]
    \centering
    \caption{Comparison of CLBG execution times (in seconds) depending on programming languages.}
    \label{fig:clbg}
    \begin{tabular}{l|*{5}r}
        \hline
                                    & \bf C       & \bf C++     & \bf Java & \bf Python     & \bf Go        \\
        \hline
        \hline
        \textsf{pidigits}           & \best{1.75} & 1.89        & 3.13     & \worst{3.51}   & 2.04          \\
        \textsf{reverse-complement} & \best{1.75} & 2.95        & 3.31     & \worst{16.76}  & 4.00          \\
        \textsf{regx-redux}         & \best{1.45} & 1.66        & 10.5     & 15.56          & \worst{28.69} \\
        \textsf{k-nucleotide}       & 5.07        & \best{3.66} & 8.66     & \worst{79.79}  & 15.36         \\
        \textsf{binary-trees}       & \best{2.55} & 2.63        & 8.28     & \worst{92.72}  & 28.90         \\
        \textsf{fasta}              & \best{1.32} & 1.33        & 2.32     & \worst{62.88}  & 2.07          \\
        \textsf{Fannkuch-redux}     & \best{8.72} & 10.62       & 17.9     & \worst{547.23} & 17.82         \\
        \textsf{n-body}             & 9.17        & \best{8.24} & 22.0     & \worst{882.00} & 21.00         \\
        \textsf{spectral-norm}      & 1.99        & \best{1.98} & 4.27     & \worst{193.86} & 3.95          \\
        \textsf{Mandelbort}         & 1.64        & \best{1.51} & 6.96     & \worst{279.68} & 5.47          \\
        \hline
    \end{tabular}
\end{table}


\subsection{Python Limits}
To reduce the energy consumption of Python, we started by targeting the main usage of this programming language, which is revealed to be data science and web development.
\Cref{fig:usecase} illustrates a study published by the JetBrain company on Python developers\footnote{\url{https://www.jetbrains.com/lp/python-developers-survey-2020}}.
57\% of the respondents reported that they use Python for data science, 51\% said they are using it for web development, and around 40\% are using it for system administration.
\note{The options in this survey were not mutually exclusive. As a result, the total of the percentages is greater than $100\%$.}


\begin{figure}[hbt]
    \centering
    \includegraphics[width=\linewidth]{imgs/python_use_cases}
    \caption{Use cases of Python (source: JetBrain).}
    \label{fig:usecase}
\end{figure}

In this chapter, we investigate the energy footprint of Python in its most popular domains of adoption.
We first explore the data and control structures, aiming to reveal some fundamental guidelines, as \citeauthor{s} did in~\cite{hasan_energy_2016-1}.
Then, we measure the energy consumption of several Python implementations to propose a non-intrusive technique to improve energy efficiency.
Therefore, this chapter will address the following research questions:
\begin{compactenum}[\indent\bf RQ\,1:]
    \item \emph{What is the energy footprint of Python when used in data science?}
    \item \emph{Are the Python guidelines energy efficient by construction?}
    \item \emph{Can we reduce the energy consumption of Python programs without altering the source code?}
\end{compactenum}

To answer these questions, we report on 4 case studies that intend to answer these research questions.
First, we study the energy behavior of Python in two application contexts: web applications (cf. \Cref{sec:webdev}) and machine learning.
Then, we dive deeper into the energy consumption of Python core structures, before concluding with the impact of the Python interpreter on energy consumption.

\section{Green Web Development}\label{sec:webdev}
Django~\footnote{\url{https://www.djangoproject.com/}} and Flask\footnote{\url{https://flask.palletsprojects.com/}}  are the most popular frameworks for web development. According to a Jetbrains poll\footnote{\url{https://lp.jetbrains.com/python-developers-survey-2021/}} , Django is used in 40\% of the cases, while Flask is used in 41\% of the cases.

In contrast to Flask, which is still a microframework, Django is a high-level web framework that provides a standard method for creating and maintaining complex and scalable database-driven websites quickly and effectively. Therefore, this study will focus on The latter one.

\subsection{life cycle of a request in Django}

Django is a MVT (Model-View-Template) framework, which means that it follows the MVC (Model-View-Controller) pattern. Figure~\ref{fig:django-life-cycle} describes the life cycle of a request in this framework.  Whenever a request arrives in Django, it is processed by several middlewares, one at a time. These middlewares are in charge of authentication, security, and so on.
Once the request is processed by these middlewares, it is passed to the URL Router, which will simply extract the URL from the request and will try to match it to the defined URLs.
After getting the matching URL, the corresponding view function is called. This function is responsible for treating the request, gathering the data, and then generating the response that will be put inside a template to be returned as a response.
However, as mentioned before, Django is an MVT model. which offers an automatic way to retrieve data from the database to the view using the \emph{Object Relational Mapping} (ORM)~\cite{o2008object}.

\begin{figure}[hbt]
    \centering
    \includegraphics[width=\linewidth]{imgs/django_request_lifecycle}
    \caption{Request-Response life cycle in Django}
    \label{fig:django-life-cycle}
\end{figure}

First, we start by investigating the energy consumption of the request-response life cycle in Django, in ordrer to determine which part consumes the most energy. To do so, we created a simple Django application that simply returns the response of the request. and then tracked its energy consumption using joulehunter~\footnote{\url{https://github.com/powerapi-ng/joulehunter}}.

Joulehunter is an open source library that we developed in order to help practitioners find the energy hotspots in their applications using statistical profilers. In the case of Django, it can be added as middleware without any extra setup or changes to the source code. The energy consumption of the request-response life cycle in Django is shown in the figure~\ref{fig:django_life_cycle_naive}.As we can notice, 91.4\% of the total energy consumption is spent to resolve the request by retrieving the data while only 5.27\% is spent to render the response.

\begin{figure}[hbt]
    \centering
    \includegraphics[width=\linewidth]{imgs/django_life_cycle_naive}
    \caption{Tree representation of the energy consumption of a single request in django (naive version)}
    \label{fig:django_life_cycle_naive}
\end{figure}



%% REMARQUE the experiment with joulehunter isn't conducted within the same machine that is used for generating the boxplot graphs 
% We have chosen Django as our web framework.
Therefore,we chose to observe if the choice of the database and the ORM impacts the energy consumption and the performance of the website.
To accomplish so, we examined the cost of a single request of the prior website using various ways to extract data from the database.
We considered using two different databases, \textsc{PostgreSQL} and \textsc{SQlite3}, that store the same records, and three different ways to fetch the records.
\begin{enumerate}
    \item \textsf{Vanilla} relies only on the ORM to retrieve the data,
    \item \textsf{Prefetch} queries the data before being requested,
    \item \textsf{Optimized} leverages SQL without passing by the ORM.
\end{enumerate}


%%%%% have to rephrase this 
As one can observe in \Cref{fig:django}, the strategy to query stored data has a huge impact on the energy consumption.
As the \textsf{Vanilla} strategy can consume up to $10\times$ more energy than the \textsf{Optimized} one.
Conversely, the choice of the database does not exhibit a key impact on the total energy despite their different behavior regarding the execution time and the average power.
This can be useful to support developers in choosing which database engine they can adopt, based on the number of expected requests and the targeted performance.

\begin{figure}[!hbt]
    \centering
    \includegraphics[width=\linewidth]{imgs/django}
    \caption{Energy behavior resulting from data access strategies.}
    \label{fig:django}
\end{figure}

Another interesting observation is the impact of the interpreter, as \Cref{fig:django} highlights.
For example, using the \textsf{Pypy} interpreter reduces the energy consumption, even when we adopting the \textsf{Vanilla} strategy.

Finally, we run the same experiment with the \textsf{Optimized} strategy 
using joulehunter
.Figure~\ref{fig:django_profiled_optimized} shows the new graph of the energy consumption. As one can notice. while the rendering method consumed the same amount of energy as in  the \textsf{Vanilla} strategy (arround 1.3KJ), the resolve part droped by $20\times$. 

\paragraph{Conclusion}
We conclude this study by stating that the database and ORM selection have a major impact on the website's performance and energy consumption.
And unlike the rendering portion, there are many options available for the database, the ORM, and the data handling strategy.
\begin{figure}[hbt]
    \centering
    \includegraphics[width=\linewidth]{imgs/django_profiled_optimized}
    \caption{Tree representation of the energy consumption of a single request in django (naive version)}
    \label{fig:django_profiled_optimized}
\end{figure}



\Cref{fig:python_multiprocessing} reports on the energy consumption of Python applications when introducing parallelism.
Python applications are single-threaded systems constrained by the \emph{global lock system} (GLS).
However, due to the increase of cores/threads per CPU, many Python libraries started to take advantage of this hardware feature by allowing concurrent execution of Python processes.
For example, the multiprocessing library\footnote{\url{https://docs.python.org/3/library/multiprocessing.html}} spawns sub-processes to increase the degree of concurrency of Python applications.
As one can see in \Cref{fig:python_multiprocessing}, the energy consumption is correlated with the number of threads until reaching the limit of physical cores, when concurrent processes start to compete for the CPU.

\begin{figure}[hbt]
    \centering
    \includegraphics[width=\linewidth]{imgs/multiprocessing_energyvstime}
    \caption{Energy consumption of Python multiprocessing depending on the number of exploited threads.}
    \label{fig:python_multiprocessing}
\end{figure}

Before reaching the limit of physical cores, we also observe that the scheduler of the operating system tends to favor the execution of processes on the same physical core, by taking advantage of the hyper-thread feature.
While this strategy aims to save energy by leveraging the ACPI P-states and C-states of unallocated cores, this leads to increase execution times.

% Another key finding is when we hit the number of physical cores. there was an increase of execution time but still reduced energy, the reason behind this is the scheduler of the operating system. Basically it favors the hyper threads than the physical core which will lead to some context switches which cause the slow behavior, however in the other hand the other two physical cores are not consuming energy, neither their hyper threads which explains the gain of the energy consumption in this case.
% in the Chapter %% reference the work of intern from lyon  
% we discuss a deeper this behavior of the scheduler and in that case we confirm that it is not related to python but it is more generic behaviour %% maybe even this graph should go there 


\subsection{Python Insights}
This section aims to formulate some actionable insights to optimize the energy consumption of Python applications without altering the source code.
Therefore, we start by extending the work of \citeauthor{hasan_energy_2016} and \citeauthor{oliveira_recommending_nodate} to the context of Python environments.




%%%% concurency 
Another field of investigation is the type of data and control structures that might impact the energy consumption.
We iterate over a list using 3 methods.
First, the classical \texttt{for(i in range(len(n)))}.
However, as we can see here, unlike other programming languages, it requires extra operations, such as determining the length of the collection and then using the iterator range.
So, we tried the more adapted version \texttt{for(element in collection)}.
Moreover, in most programming languages. the \texttt{for} loop is translated to a \texttt{while} loop ( transformation from D type to B type -- asm -- ), therefore we wanted to compare this with a \texttt{while} version.
After determining the main ways to iterate over a \texttt{loop}, we run the collection algorithms of different data types to observe if they impact the energy consumption of the code, and we repeated it for the size of the collection.

As one can see in \Cref{fig:pythonloops}, the type of the data has no impact on the energy consumption.
However, the way one iterates over the collection has a huge factor.
Interestingly, the \texttt{for in range} loop was by far the optimal one, followed by the regular \texttt{for in collection}, and the \texttt{while} part was the last one with an overhead of 400\% compared to the first option.

\begin{figure}
    \centering
    \includegraphics[width=\linewidth]{imgs/python_iterations}
    \caption{Comparison of the energy consumption of different Python loops.}
    \label{fig:pythonloops}
\end{figure}

%%% putting the idea, later i ll reformulate it 
The reason behind such a behavior is mainly related to how the Python interpreter is implemented\footnote{\url{https://www.python.org/doc/essays/list2str}},\footnote{\url{https://www.pythonpool.com/for-vs-while-loop-python/}}.  
To reduce the latency of Python applications, most of the built-in functions and operations are written in C, and the same goes for the function \texttt{range}.\footnote{\url{ http://python-history.blogspot.com/2010/06/from-list-comprehensions-to-generator.html?m=1}}
Furthermore, the function \texttt{len} has a complexity of $\mathcal{O}(1)$ as it is based on the function \texttt{Py\_SIZE} of C, which stores the length in a field for the object~\footnote{\url{https://wiki.python.org/moin/TimeComplexity}}.
Therefore, the \texttt{for in range} is creating a new iterator that has the same length as the first one and, for each iteration, requires a second access (\texttt{l[i]}) instead of one---explaining the doubled time.
The \texttt{while} is even slower due to the implicit increment of the variable, which causes an extra operation during the loop.
To confirm this hypothesis, we tried to construct a new list by editing the elements of the previous one (cf. \Cref{fig:pythontreatement}).
And, as predicted, the built-in methods are the most energy-saving ones, while the customize \texttt{while} loop is the heaviest.

\begin{figure}
    \centering
    \includegraphics[width=\linewidth]{imgs/python_treatemens}
    \caption{Comparison of the energy consumption of different methods to convert a list.}
    \label{fig:pythontreatement}
\end{figure}

Another interesting finding is the impact of anonymous functions (also known as lambda expressions) on energy consumption.
The reason is that Python treats these functions as local variables, unlike the predefined ones which are global in our case.
Therefore, they are faster and consume less energy.  % same reference https://www.python.org/doc/essays/list2str/ 


\paragraph{Conclusion}
This study demonstrates that the optimal way to reduce the energy consumption of Python application is to follow the guidelines and to privilege the built-in functions.


\subsection{Python \& Multiprocessing}
\Cref{fig:python_multiprocessing} compares the behaviour of python programs when we try to introduce the parallelism.
As we know python is a single threaded program thanks to ghe GLS (Global Lock System) howerver due to the increase of the number of cores /threads per cpu it many libraries stared to take advanteage of this feature so most of them they will try to simulate the multithreading by using multiple instances of the processor % TODO: TO VERIFY 

\begin{figure}[hbt]
    \centering
    \includegraphics[width=\linewidth]{imgs/multiprocessing_energyvstime}
    \caption{energy behaviour based on multiprocessing}
    \label{fig:python_multiprocessing}
\end{figure}

As we can see in the graph the energy consumption is correlate with the number of threads until we arrive to the limit of the cores and then we lose the adventage of the multiprocessing , well in that case we pass from parallelism to concurency. where different sub process have to compete for the CPU Ressources.
Another finding is when we hit the number of physical cores. there was an increase of execution time but still reduced energy, the reason behind this is the scheduler of the operating system. basicaly he favorits the hyper threads than the physical core wich will lead to some context switches which cause the slow behaviour, howerver in the other hand the other two physical cores are not consuming energy, neither their hyper threads which explains the gain of the eneryg consumption in this case.
in the Chapter %% reference the work of intern from lyon  
we discuss a deeper this behaviour of the scheduler and in that case we confirm that it is not related to python but it is more generic behaviour %% maybe even this graph should go there 

%% todo 
% add performance 
% add dram \ a glimps 

\subsection{Python \& Machine Learning}
Machine learning is becoming an integral part of our daily lives, growing more potent and energy-hungry each year.

As machine learning can have a significant impact on climate change, it is vital to investigate mitigation techniques.

\paragraph{Hardware settings}
Chifflot 8 from Grid 5000's Lille site was used for all of the trials.
The machine is outfitted with two Intel Xeon Gold 6126 CPUs, each having 12 physical cores, 192 GB of RAM, and two 32 GB Tesla V100 GPUs.

\paragraph{Software settings}
For the sake of reproducibility, each experiment is done within a Docker container using Jupyter lab. These tests are run on top of a minimal version of Debian-10 to increase the accuracy of the tests by eliminating any unnecessary processes.


\subsubsection{Input Workload}
\paragraph{Models}
Several models were developed, however, only two were used in the final trials because they achieved 94 percent accuracy in an acceptable length of time.
David Page's cifar10-fast~\footnote{\url{https://github.com/davidcpage/cifar10-fast}} and Woonhyuk Baek's torch skeleton~\footnote{\url{https://github.com/wbaek/torchskeleton}} are shown here.

\paragraph{Datasets}
The CIFAR-10 dataset was the major source of data for the studies.
It is made up of 60000 32x32 color images grouped into ten categories.

Some experiments were done using the MNIST dataset of handwritten digits to validate the results acquired from the first dataset.
The model did not need to be updated because the 60000 28x28 grayscale photos were padded.


\subsubsection{candidates}
The experiments were run with several different CPU and GPU configurations:
\begin{itemize}
    \item with and without GPU,
    \item with and without CPU hyper-threading,
    \item different number of CPU physical cores.
\end{itemize}

\subsubsection{Key Performance Metrics}
We used Pytorch 1.10.0 to train those models and pyjoules to measure the energy consumption of the GPU and CPU.
\begin{itemize}
    \item accuracy : in \%
    \item execution time : in seconds for both the duration of each epoch and the total duration to a chieve a certain accuracy
    \item total energy consumption : for the CPU and the  GPU
    \item
\end{itemize}


\section{Results \& Findings}
As the model's accuracy increases, so does the energy required for the next accuracy increment.

Figure \ref{fig:cum_energy_vs_accuracy} depicts how the curve steepens as training progresses.
For example, training to 90\% accuracy requires three times the energy required for training to 80\% accuracy.
\begin{figure}
    \centering
    \includegraphics[width=\linewidth]{imgs/accuracy_basedonepoch}
    \caption{accuracy based on epoch  }
    \label{fig:p2}
\end{figure}

\begin{figure}
    \centering
    \includegraphics[width=\linewidth]{imgs/cumulative_energy_vs_accuracy}
    \caption{cumulative energy consumption vs accuracy
    }
    \label{fig:cum_energy_vs_accuracy}
\end{figure}
\begin{figure}
    \centering
    \includegraphics[width=\linewidth]{imgs/power_gpu_baedonepoche}
    \caption{evolution of average gpu power based on epoch  }
    \label{fig:p2}
\end{figure}
\begin{figure}
    \centering
    \includegraphics[width=\linewidth]{imgs/cumulative_energy_fast10}
    \caption{cumulative energy of fast10 benchmark within different configurations }
    \label{fig:p2}
\end{figure}

\paragraph{Conclusion}
We discovered that when a model's accuracy improves, so does the energy required for a subsequent accuracy gain.
This begs the question of when we should discontinue training.
Is a 10\% increase in accuracy worth it if we have to spend three times the energy?
Even while utilizing a GPU consumes more total power, the training time is significantly reduced.
As a result, the utilization of a GPU is required to lower the energy consumption of the training.
Some experiments revealed unusual behavior that was not studied during this internship.
These could be examined in future studies and perhaps used to achieve more energy savings.
The execution time did not depend on the number of cores employed in several studies.
This parameter had a significant impact on training time in others.
As a result, the best core configuration will be determined by the script.
The next steps in this area could include exploring the reasons for these discrepancies and possibly creating a script that can automatically find the ideal core configuration for a specific script.


\section{Python Interpreters}\label{sec:pythoninterpreters}
%%% note : add this to the section of python interpreters 
Due to the lack of support for most non-conventional Python interpreters, we mainly focus on micro-benchmarks.
Except for \textsc{PyPy}, most of the Python implementations do not support extra Python libraries, despite those extra implementations being developed to optimize a specific library, such as \textsf{Numba} with \textsf{Numpy}, or \textsf{intelpython} with machine learning algorithms.

\paragraph{Preliminary studies}
For the first studies, we used the official version of Python, because the goal was mainly to highlight the impact of the structure of code on energy consumption.
One main drawback of the previous method is the work to be done to update the existing code base to reduce  energy consumption.
To avoid such hustle, we tried to find a non-intrusive approach to make the Python code more eco-friendly without altering its structure.
Python is an interpreted language, which led many initiatives to implement their own interpreter to improve one or many aspects of the Python code.
In the following section, we discuss the impact of those implementations on the energy consumption of python programs, and in which case, one should use a non-conventional interpreter to save the energy consumption of their application.

To do so, we gathered a list of interpreters, transpilers and other optimization libraries that can contribute to reduce the energy consumption of legacy Python applications:
\begin{enumerate}
    \item \textbf{CPython}:\footnote{\url{https://www.python.org/}}
          This Python interpreter, written in C, is the reference interpreter of Python.
          CPython compiles the source code into byte-code and then interprets it.
          The CPython project supports both versions of Python 2 and 3;
    \item \textbf{PyPy}:\footnote{\url{http://pypy.org}}
          An alternative implementation of the Python interpreter.
          It is written using \emph{RPython} to use the JIT.
          It compiles the most used portions of the Python code into a binary code for better performance.
          To benefit from these optimizations, the program has to be executed for at least for few seconds so the JIT has enough time to warm up, the JIT optimization are only applied to the code written by the developer and not to external libraries;
    \item \textbf{Cython}:\footnote{\url{https://github.com/cython/cython}}
          A static compiler for Python.
          It translates the Python code into C, and then compiles it using a C compiler.
          It also supports an extended version of the Python language that allows programmers to call \emph{C functions}, declare \emph{C types} and use static types, which will help the translation of Python objects into native types, such as integers, float.
          This often means better performances, since native C libraries are almost all the time faster than the Python written once~\cite{pereira_energy_2017};
    \item \textbf{Intel\,Python}:\footnote{\url{https://software.intel.com/en-us/distribution-for-python}}
          A customized interpreter developed by Intel to enhance performances of Python programs.
          It is dedicated to data sciences and high-performance computing.
          It uses some Intel kernel libraries, such as Math Kernel Library (Intel MKL\footnote{\url{https://software.intel.com/en-us/mkl}}) and data analytics acceleration library (Intel DAAL\footnote{\url{https://software.intel.com/en-us/intel-daal}}).
          It supports both versions of Python;
    \item \textbf{Active\,Python}:\footnote{\url{https://www.activestate.com/products/activepython/}}
          It is developed by the Activestates company and provides a standardized Python distribution to ensure license compliance, security, compatibility and performance.
          Therefore, ActivePython implements its built-in packages (more than 300 packages) and supports both versions of Python;
    \item \textbf{IronPython}:\footnote{\url{https://ironpython.net}}
          A .Net-based Python interpretation platform written in C\# that is used with the .Net virtual machine or Mono.
          It benefits from all the optimizations of .Net virtual machines, such as the JIT and garbage collector mechanisms;
    \item \textbf{GraalPython}:\footnote{\url{https://github.com/graalvm/graalpython/}}
          A Python interpreter that is based on GraalVM\footnote{\url{https://www.graalvm.org/docs/why-graal/}} (a universal virtual machine developed by oracle for running applications written in different programming languages).
          For the time being, it only supports Python~3 and it is still in the experimental stage;
    \item \textbf{Jython}:\footnote{\url{https://jython.github.io}}
          An implementation of Python programming language written in Java for the \emph{Java Virtual Machine} (JVM).
          Similar to IronPython and GraalPython, it leverages the optimization mechanisms provided by the JVM to enhance the Python performances;
    \item \textbf{MicroPython}:\footnote{\url{http://micropython.org}}
          A lightweight Python version dedicated to embedded systems and micro-controllers;
    \item \textbf{Nuitka}:\footnote{\url{http://nuitka.net/pages/overview.html}}
          A Python compiler written in Python that generates a binary executable from Python code.
          It translates the Python code into a C program that is then compiled into a binary executable;
    \item \textbf{Numba}:\footnote{\url{https://numba.pydata.org}}
          A library that includes JIT compiler to enhance the performances of Python functions using the industry-standard LLVM compiler library;
    \item \textbf{Shedskin}:\footnote{\url{https://github.com/shedskin/shedskin}}
          A static transpiler that translates implicitly statically typed python into C++ code;
    \item \textbf{Hope}~\cite{akeret_hope_2015}:
          A Python library that aims to introduce JIT compiler into the Python code;
    \item \textbf{Parakeet}~\cite{DBLP:conf/hotpar/RubinsteynHWS12}:
          A runtime accelerator for an array-oriented subset of Python;
    \item \textbf{Stackless\,Python}:\footnote{\url{https://github.com/stackless-dev/stackless/wiki}}
          An interpreter that focuses on enhancing multi-threading programming;
    \item \textbf{Pyjion}:\footnote{\url{https://github.com/microsoft/pyjion}}
          A JIT API for CPython, same purpose as Parakeet and Hope;
    \item \textbf{Pyston}:\footnote{\url{https://blog.pyston.org}}
          A performance-oriented Python implementation built using LLVM and modern JIT techniques.
          The project is funded by Dropbox;
    \item \textbf{Grumpy}:\footnote{\url{https://github.com/google/grumpy}}
          A source-to-source transpiler that translates the Python code into Go before being compiled to a binary executable.
          It also offers an interpreter, called \emph{grumprun}, which can directly execute the Python code.
          Unfortunately, we cannot use it because the project is already outdated (last commit is in 2017) and it has a lot of limitations in terms of supporting the Python language, such as some built-in functions and standard libraries;
    \item \textbf{Psyco}:\footnote{\url{http://psyco.sourceforge.net}}
          A JIT compiler for Python;
    \item \textbf{Unladen\,Swallow}:\footnote{\url{https://unladen-swallow.readthedocs.io/en/latest/}}
          An attempt to (use) LLVM as a JIT compiler for CPython.
\end{enumerate}


\subsection{Runtime Classification}
Before further proceeding with the list of candidate runtime for Python applications, we propose a classification according to several criteria:
\begin{description}
    \item[\bf Type] refers to the category of runtime infrastructure that supports the execution of a Python application.
        In particular, we consider 3 types of environments: \emph{Interpreter}, \emph{Compiler} and \emph{Library};
        \emph{Interpreter} refers to the class of environment that does not require any preprocessing of Python source code;
        \emph{Compiler} introduces a compilation phase before the execution of the application.
        Finally, \emph{Library} requires some modification of the source code;
    \item[\bf Runtime] refers to the technology supporting the execution of a Python application.
        This technology can refer to the programming language used to program the interpreter, the target language for a compiler or a library;
    \item[\bf JIT optimization] refers to the support of \emph{just-in-time} compilation in the runtime infrastructure supporting the execution of the application;
    \item[\bf GC optimization] refers to the support of \emph{garbage collection} in the runtime infrastructure supporting the execution of the application;
    \item[\bf Python version(s)] refers to the list of Python source code versions supported by the runtime environment.
\end{description}

We should explain the classification of these runtimes
\begin{table}
    \caption{Classification of Python implementations}
    \label{fig:python-classes}
    \center
    \begin{tabular}{|l|c|c|c|c|c|c|}
        \hline
        \multirow{2}{*}{\bf Name} & \multirow{2}{*}{\bf Type} & \multirow{2}{*}{\bf Runtime} & \multicolumn{2}{c|}{\bf Optimisations} & \multicolumn{2}{c|}{\bf Python}                     \\
        % \cline{1-5} 
                                  &                           &                              & {\bf JIT}                              & {\bf GC}                        & {\bf 2} & {\bf 3} \\
        \hline
        \hline
        CPython                   & Interpreter               & C                            & \no                                    & \no                             & \yes    & \yes    \\
        \hline
        Intel\,Python             & Interpreter               & C                            & \no                                    & \no                             & \yes    & \yes    \\
        \hline
        ActivePython              & Interpreter               & C                            & \no                                    & \yes                            & \yes    & \yes    \\
        \hline
        PyPy                      & Interpreter               & Python                       & \yes                                   & \yes                            & \yes    & \yes    \\
        \hline
        IronPython                & Interpreter               & .Net                         & \yes                                   & \yes                            & \yes    & \yes    \\
        \hline
        GraalPython               & Interpreter               & GraalVM                      & \yes                                   & \yes                            & \no     & \yes    \\
        \hline
        Jython                    & Interpreter               & Java                         & \yes                                   & \yes                            & \yes    & \no     \\
        \hline
        Stackless\,Python         & Interpreter               & Python                       & \no                                    & \no                             & \yes    & \no     \\
        \hline
        MicroPython               & Interpreter               & c                            & \no                                    & \no                             & \no     & \yes    \\
        \hline
        Pyston                    & Interpreter               & LLVM                         & \yes                                   & \no                             & \yes    & \no     \\
        \hline
        Unladen\,Swallow          & Interpreter               & LLVM                         & \yes                                   & \no                             & \yes    & \no     \\
        \hline
        \hline
        Cython                    & Compiler                  & C                            & \no                                    & \no                             & \yes    & \yes    \\
        \hline
        Nuitka                    & Compiler                  & C                            & \no                                    & \no                             & \yes    & \yes    \\
        \hline
        Shedskin                  & Compiler                  & C++                          & \no                                    & \no                             & \yes    & \yes    \\
        \hline
        Grumpy                    & Compiler                  & Go                           & \no                                    & \no                             & \yes    & \yes    \\
        \hline
        \hline
        Numba                     & Library                   & C                            & \yes                                   & \no                             & \yes    & \yes    \\
        \hline
        Hope                      & Library                   & Python                       & \yes                                   & \no                             & \yes    & \yes    \\
        \hline
        Psyco                     & Library                   & Python                       & \yes                                   & \no                             & \yes    & \yes    \\
        \hline
        Pyjion                    & Library                   & .NET Core                    & \yes                                   & \no                             & \yes    & \yes    \\
        \hline
        Parakeet                  & Library                   & C                            & -                                      & \no                             & \yes    & \no     \\
        \hline
    \end{tabular}
\end{table}



\begin{table}[hbt]
    \caption{Classification of Python implementations}
    \label{fig:pythonimplementations}
    \center
    \begin{tabular}{|l|c|c|c|}
        \hline
        Version  & Interpreter     & Transpiler/Compiler & Jit library \\
        \hline
        \hline
        Python~2 & Cpython2        & Cython2             & Numba 2     \\
        % \cline{2-4} 
                 & Pypy2           & Shesdskin           & Hope        \\
        % \cline{2-4} 
                 & Pytson          & Grumpy              & Parakeet    \\
                 & Ironpython      &                     & Psyco       \\
                 & Jython          &                     & Pyjion      \\
        % &  &  &  \\
                 & Micropython     &                     &             \\
                 & Pysec           &                     &             \\
                 & StacklessPython &                     &             \\
        % \cline{2-4} 
        \hline
        Python~3 & Cpython3        & Nuitka              & Numba3      \\
                 & Pypy3           &                     &             \\
                 & GraalPython     &                     &             \\
        \hline
    \end{tabular}
\end{table}

There are other implementations that we did not consider because either the project aborted many years ago or it has very limited support for Python features.
After the collection of those implementations, we filtered them. To keep only the versions that are still maintained and support most Python features.
and we classified them into 3 categories depending on their integration with the python code.
In \Cref{fig:pythonimplementations}, we describe the implementations that we kept and the version of each implementation and its category.

\begin{figure}
    \centering
    \includegraphics[width=\linewidth]{imgs/python-implementations-tree}
    \caption{Python interpeters}
    \label{fig:interpreters}
\end{figure}



\section{experimetal protocole}

As we have discussed in the previous chapter \ref{chapter:benchmarking}. instead of running the tests, the idea was to design a system that allows practiouners to reproduce and extends our tests. and then we use the same system to run answer some of researchs question.
%%%% probably ill resue the same things from other chapters 
\subsection{measurement context}
\paragraph{Hardware settings}

all our tests have been executed in a Dell PowerEdge C6420 server machine. A summary of its hardware is listed in table \ref{fig:dahuconfig}. The machine is equiped with a minimal version of Debian\,9 (4.9.0 kernel version) where we install Docker (version 18.09.5).



\begin{table}[hbt]
    \begin{tabular}{ll}
        \hline
        CPU     & Intel Xeon Gold 6130 (Skylake, 2.10GHz, 2 CPUs/node, 16 cores/CPU)                                                     \\
        Memory  & 192 GiB                                                                                                                \\
        Storage & 240 GB SSD SATA Samsung MZ7KM240HMHQ0D3                                                                                \\
                & 480 GB SSD SATA Samsung MZ7KM480HMHQ0D3                                                                                \\
                & 4.0 TB HDD SATA Seagate                                                                                                \\
        Network & eth0/enp24s0f0, Ethernet, configured rate: 10 Gbps, model: Intel Ethernet Controller X710 for 10GbE SFP+, driver: i40e \\
                & ib0, Omni-Path, configured rate: 100 Gbps, model: Intel Omni-Path HFI Silicon 100 Series [discrete], driver: hfi1      \\
        \hline
    \end{tabular}
    \caption{Testing Machine Configuration }
    \label{fig:dahuconfig}
\end{table}

\begin{figure}[hbt]
    \includegraphics[width=\linewidth]{imgs/SmartWatts.png}
    \caption{powerapi architecture}
    \label{fig:powerapi}
\end{figure}

\paragraph{software settings}
For the sake of reproducibility, each experiment runs within a Docker container.
% for each test we create a docker image.
\subsection{Metrics}
Our focus will be mainly on CPU energy consumption because it is ten folds more than the DRAM one, since it is finit job benchmarking, time is highly correlated within the energy, and it will be only useful to explain certain energytical behaviour so we wont put a lot of focus on this metric.

%% basically i what i want to say is that we we care mostly about energy, because there are studies that were made on the performance, plus the DRAM is not that interested even within the heavy memory  benchmakrs 
% TODO : ADD DRAM graphs for vectors and matrix multiply 


\paragraph{Energy measurement}

As we know, the energy of a program is the integrale of its power overtime. For us ower case, we used Intel \emph{Running Power Average Limit} (RAPL)~\cite{Khan:2018:RAE:3199681.3177754} to collect the power samples of the running tests,

We used used \textsc{PowerAPI}~\cite{DBLP:journals/jss/ColmantRKSFS18}, to report Data collected by intel Rapl and send it into another machine that we call computing machine. then we calculate the Energy using  the the trapezoidal rule.


\ref{fig:trapezrule}
\begin{figure}[hbt]
    \centering
    \begin{equation}
        E = \int^a_b P(t)dt \simeq \sum^n_{k=1} \frac{P(t_k-1)+P(t_k)}{2}
    \end{equation}
    \caption{}
    \label{fig:trapezrule}
\end{figure}



figure \ref{fig:powerapi} shows the architecture of our testing model.


The reason of separation between data collection and energy calculation is to minimise any interference with the test so our sensor is a light c program running inside a docker container.
\subsection{tests preparation}



To study the behaviour of the python implementation regarding the energy consumption, we have to focus on the effect of the implementation and mitigate the maximum any side effects such as the organisation of the code or any extra consumption due to the operating system or tier libraries.
Therefore, for each test we took the implementation written in python2 as a reference and tried to use it in other implementations as it is. If it is not supported by python3, we transformed the code using the officiel library  \emph{2to3}\footnote{\url{https://docs.python.org/3.7/library/2to3.html}}.
In the case of the libraries that use \emph{JIT} adding a decorator to the function that we want to optimize was enough, if there are other changes we assume that they alter the original code which is against our purpose.

Each test is implemented in a Docker container for the several reasons
\begin{itemize}
    \item Isolation: each container has only the test program implemented with a single python runtime to remove any interference between different implementations.
    \item Deplpoyment: to use the testing machine without extra configurations that may alter the behaviour of the os toward the enrgy consumtion
    \item Reproductibility: One of the most frequent benchmark crimes~\cite{DBLP:journals/corr/abs-1801-02381} in research is the lack of Reproductibility, by using Docker we ensure that each test has an Image that will be accessible in public.
\end{itemize}

Despite the presence of the official docker images for the most of the runtimes, we prefered to build our own using the same reference image in order to remove any bias due to the Os used in the offical image.We used ArchLinux with the kernel version 4.9.184 as a base image.


\subsection{Extension}
As we have done with the previous chapters. we provide a tool that allows to extend the tests with new workloads and new candidates.
In the repository \footnote{\url{https://github.com/chakib-belgaid/python-implementations}}.
we have a tool that allows to generate new workloads and new candidates.
The script \texttt{generator.py} allows to create new benchmarks by implementing a python code within different interpeters. then it generates \texttt{launcher-benchmark.sh} that can be used directly to run the expermient. Furthermore all the successful implementations are stored in seperate directory and the that couldn't work ( mostly because of compatibility issues) stored in a recap file called \texttt{benchmarkTest.md}. where benchmark is the name of the new workload;
To add extra \textbf{candidates} one should add a base docker file
that containes the new implementation and if there should be extra manipulation that should be added to the workload files ,such as adding new decorator or changing some parameters, then they should be added as an extra function in the script \texttt{generator.py}. Finally they should be included in the python candidates.

\section{Results and finding}

\subsection{Preliminary stduies}
This part will be dedicated to analyse the initial behaviour of different python interpereters.
First we started with a typical benchmark \em binarrytree. where we compare the energy consumption of the different implementations.


\section{placeholder}

we performed a shapiro normality test for the first on the results, and almost all the p-values smaller than alpha=0.01\% therefore the distrubution is not normal

for arrays and vectors  graalpython took so much time that we decided to remove it

the startup cost for ironpython was too high
for graalpython and ipy they couldnt handle big numbers hence the overflow  when it comes to execution time
but since we measure the energy outside the values of the energy aren't impacted

as usual there is  correlaction between DRAM and CPU so no need to classify based on DRAM + the DRAM consumes less 10\% of the  energy compared to the CPU

so the plan is to prove that there is a statistical difference
then we use the multiparameter optimization , howerver we stop in the phase where we calculate the score
because we dont know the weight of each parameter ( of tommti microbenchmark) and instead of doing the sum with the coefissients we use the radar plot to let the reader decide which one is the most suitable for his usage. howerver since the differences are gigantics we change the scale to logarithmic to be easier for the eye to read



\begin{figure}

    \label{table:bitops}
    \begin{tabular}{|lrrrrr|}
        \toprule
        benchmark    & A       & GG      & LL      & Or      & XOR     \\
        \midrule
        activepython & 676.980 & 763.208 & 651.783 & 743.016 & 728.828 \\
        cpython2     & 441.082 & 435.886 & 430.846 & 415.247 & 419.081 \\
        cpython3     & 595.209 & 685.085 & 563.839 & 657.972 & 655.560 \\
        cython       & 35.077  & 182.688 & 274.177 & 34.868  & 34.504  \\
        nuitka       & 33.260  & 32.980  & 33.256  & 33.472  & 33.030  \\
        numba2       & 9.102   & 8.411   & 9.460   & 9.375   & 9.755   \\
        numba3       & 9.566   & 10.144  & 9.219   & 9.344   & 9.665   \\
        pypy2        & 8.456   & 7.844   & 8.286   & 8.138   & 7.952   \\
        pypy3        & 7.552   & 8.093   & 8.108   & 8.669   & 8.623   \\
        shedskin     & 8.024   & 8.070   & 8.399   & 8.126   & 8.277   \\
        \bottomrule
    \end{tabular}
    \vfill
    \centering
    \includegraphics[width=\linewidth]{imgs/bitopts_mean}
    \caption{energy consumption of different impelementations using Bit Operation benchmarks (Joule) }
\end{figure}


% \begin{figure}
%     \centering
%     \includegraphics[width=\linewidth]{imgs/barplot_binarry_tree}
%     \caption{energy behaviour of different python implementations for the binarry Tree benchmark}
%     \label{fig:binarrytree}
% \end{figure}

\begin{figure}

\end{figure}
\begin{figure}
    \centering
    \includegraphics[width=\linewidth]{imgs/dendogram_interpreters}
    \caption{Dendogram of the different implementations}
    \label{fig:dendogram}
\end{figure}
\begin{figure}
    \centering
    \includegraphics[width=\linewidth]{imgs/alltomti_performance}
    \caption{different interpererts optimisation }
    \label{fig:tommi_all}
\end{figure}
\begin{figure}
    \centering
    \includegraphics[width=\linewidth]{imgs/tommti_compare__pypy2_pypy3}
    \caption{green factor of pypy }
    \label{fig:pypy2vspypy3}
\end{figure}

\begin{figure}
    \centering
    \includegraphics[width=\linewidth]{imgs/tommti_compare__cpython3_numba3_pypy3}
    \caption{comparaison of pypy vs python vs numba }
    \label{fig:p3}
\end{figure}

\begin{figure}
    \centering
    \includegraphics[width=\linewidth]{imgs/tommti_compare__cpython2_pypy2_numba2}
    \caption{comparaison of pypy vs python vs numba }
    \label{fig:p2}
\end{figure}


As we can see in figure \ref{fig:tommi_all}, there is no evolution between cpython2 and cpython3
Intelpython and active python both folows the same behaviour. one can conclude that the work that have been on those interpreters is mainly
to improve a specific purpose, Active python claims that their version is focused on secuirty which explains the lack of some performances due to the introduction of more reonforcment
.Morver Intel published their version of python as a dedicated for machine learning. Unfortunately the tommti benchmark is a set that focuses on the general purpose programming
which does not reflect the performance of the machine learning.
Another aspect of a such behaviour might be due to the fact of the processors that were used in the testbed. and it might change in the future if we include the GPU part for the machine learning benchmarks
% reference the work of the intern 
As for nuitka. there were no optimization in the energy consumpotion despite the fact that is it a complier.
However if we dig through the nuitka mechanisms. they basically embed the python code with an interpreter.
Unllike nuitka shedskin exhebit a the best energy consumption pattern when it comes to the arithmetic operartions. One can conclude it is due to the fact of the native type of the variables. unlike the interpreters where they are treated as object in the begining.

for the other interpeters pypy is very promising especially when it comes to data maniupulation as one can see in the figure \ref{fig:bar_tommtei_vector}
pypy is by far the best interperer when it comes to treating vectors.
numba2 introduced the JIT but wasn't as promising as numba3.

for the other vm based interpereters. jython and ipy lacked in term of energy optimisation which was kinda expected since they were in their the begining of the stage and the main puprose of such implementation is to link the bytecode generated by jython and ironpython with their respectives virtual machin.

Unlike the the previous interpeters. graal exhebits a certin promises when it comes to complex algorithms - nested loops -
micro python is dedicated to embeded systems so lunching it is powerful cpu machines will be misleading.

Most of the interperters had the same behaviour when it comes to the input outputs. except for jython which was kinda of a anomaly probably due to the lack of the optimizations.
% pypy handles exceptions very well 


\section{conclusion}
One may observe that the choice of Python interpreter has a significant impact on the programs' energy consumption.
This investigation is made more intriguing by the absence of a universal solution.
The primary downside is the incompatibility of some of these solutions, which causes us to make concessions when we need a generic answer.



\begin{table*}

    \caption{Energy consumption python environment}
    \label{table:GC}
    \resizebox*{\linewidth}{!}{
        \begin{tabular}{l|rr|rr|rr|rr|rr|rr}
            \toprule

            Implementation & \multicolumn{2}{c}{array} & \multicolumn{2}{c}{intArithmetic} & \multicolumn{2}{c}{doubleArithmetic} & \multicolumn{2}{c}{hashes} & \multicolumn{2}{c}{heapsort} & \multicolumn{2}{c}{trig}                                                                                         \\
            \midrule

                           & energy (J)                & \em p-values                      & energy (J)                           & \em p-values               & energy (J)                   & \em p-values             & energy (J)     & \em p-values & energy (J) & \em p-values & energy (J) & \em p-values \\
            activepython   & 4e02                      & 0.0079                            & 6.9e02                               & 0.0022                     & 6.6e02                       & 0.0043                   & 9.7e02         & 0.0079       & 2.9e02     & 0.0079       & 5.1e02     & 0.0079       \\
            cpython2       & 3.4e02                    & 0.0079                            & 5.6e02                               & 0.0022                     & 5.5e02                       & 0.0043                   & 6.5e02         & 0.0079       & 2.9e02     & 0.0079       & 4.1e02     & 0.0079       \\
            cpython3       & 3.2e02                    & 0.0079                            & 7.4e02                               & 0.0022                     & 7.3e02                       & 0.0043                   & 7.9e02         & 0.0079       & 2.4e02     & 0.0079       & 4.1e02     & 0.0079       \\
            graalpython    & 1.2e05                    & 0.0079                            & 24                                   & 0.0022                     & 24                           & 0.0043                   & 6.4e02         & 0.0079       & 1.3e05     & 0.0079       & 72         & 0.0079       \\
            intelpython2   & 3.7e02                    & 0.0079                            & 5.9e02                               & 0.0022                     & 5.6e02                       & 0.0043                   & 7.1e02         & 0.0079       & 2.7e02     & 0.0079       & 4.2e02     & 0.0079       \\
            intelpython3   & 3.5e02                    & 0.0079                            & 7.7e02                               & 0.0022                     & 7.6e02                       & 0.0043                   & 9.4e02         & 0.0079       & 2.6e02     & 0.0079       & 4.8e02     & 0.0079       \\
            ipy            & 3e02                      & 0.0079                            & 4.4e02                               & 0.0022                     & 4.7e02                       & 0.0079                   & 1.3e03         & 0.0079       & 2.6e02     & 0.0079       & 4.5e02     & 0.0079       \\
            jython         & 5.2e02                    & 0.0079                            & 1.3e02                               & 0.0022                     & 1.6e02                       & 0.0043                   & 6.4e02         & 0.0079       & 4.6e02     & 0.0079       & 6.2e02     & 0.0079       \\
            micropython    & 3.1e02                    & 0.0079                            & 8.2e02                               & 0.0022                     & 8.4e02                       & 0.0079                   & 9.5e03         & 0.0079       & 3.4e02     & 0.0079       & 5.3e02     & 0.0079       \\
            nuitka         & 2.9e02                    & 0.0079                            & 5.4e02                               & 0.0022                     & 5.5e02                       & 0.0079                   & 9.5e02         & 0.0079       & 2.2e02     & 0.0079       & 3.9e02     & 0.0079       \\
            numba2         & 1.9e02                    & 0.0079                            & 27                                   & 0.0022                     & 36                           & 0.0079                   & 6.8e02         & 0.0079       & 2.1e03     & 0.0079       & 4.4e02     & 0.0079       \\
            numba3         & \best{11}                 & \best{best}                       & 10                                   & 0.0022                     & 10                           & 0.0079                   & 9.1e02         & 0.0079       & 7.2e02     & 0.0079       & 4.5e02     & 0.0079       \\
            pypy2          & 16                        & \same{0.15}                       & 29                                   & 0.0022                     & 30                           & 0.0043                   & \best{ 1.1e02} & \best{best}  & \best{20}  & \best{best}  & 64         & 0.0079       \\
            pypy3          & 13                        & \same{0.15}                       & 17                                   & 0.0022                     & 18                           & 0.0079                   & 1.9e02         & 0.0079       & \best{20}  & 0.31         & 65         & 0.0079       \\
            shedskin       & 47                        & 0.0079                            & \best{7.4}                           & \best{best}                & \best{7.3 }                  & \best{best}              & 1.1e03         & 0.0079       & 44         & 0.0079       & \best{7.7} & \best{best}  \\
            \bottomrule
        \end{tabular}

    }
    \vfill
    \resizebox*{\linewidth}{!}{
        \begin{tabular}{l|rr|rr|rr|rr|rr|rr}

            Implementation & \multicolumn{2}{c}{longArithmetic} & \multicolumn{2}{c}{matrixMultiply} & \multicolumn{2}{c}{io} & \multicolumn{2}{c}{stringConcat} & \multicolumn{2}{c}{nestedLoop} & \multicolumn{2}{c}{except}                                                                                     \\
            \midrule
                           & energy (J)                         & \em p-values                       & energy (J)             & \em p-values                     & energy (J)                     & \em p-values               & energy (J) & \em p-values & energy (J) & \em p-values & energy (J) & \em p-values \\
            activepython   & 6.7e02                             & 0.0079                             & 4.3e02                 & 0.0079                           & 2.1e02                         & 0.0079                     & 14         & 0.0079       & 4.1e02     & 0.0079       & 2.6e02     & 0.0079       \\
            cpython2       & 5.5e02                             & 0.0079                             & 4e02                   & 0.0079                           & 2e02                           & 0.0079                     & 12         & 0.0079       & 4.2e02     & 0.0079       & 4.3e02     & 0.0079       \\
            cpython3       & 7.4e02                             & 0.0079                             & 4.4e02                 & 0.0079                           & 2e02                           & 0.0079                     & 13         & 0.0079       & 3.8e02     & 0.0079       & 2.3e02     & 0.0079       \\
            graalpython    & 24                                 & 0.0079                             & 45                     & 0.0079                           & 7.5e02                         & 0.0079                     & 26         & 0.0079       & 12         & 0.0079       & 1.6e02     & 0.0079       \\
            intelpython2   & 5.7e02                             & 0.0079                             & 4.8e02                 & 0.0079                           & 2e02                           & 0.0079                     & 13         & 0.0079       & 4.4e02     & 0.0079       & 4.6e02     & 0.0079       \\
            intelpython3   & 7.7e02                             & 0.0079                             & 5e02                   & 0.0079                           & 2.2e02                         & 0.0079                     & 14         & 0.0079       & 4.4e02     & 0.0079       & 2.7e02     & 0.0079       \\
            ipy            & 4.7e02                             & 0.0079                             & 4.1e02                 & 0.0079                           & 3.8e02                         & 0.0079                     & 49         & 0.0079       & 3.3e02     & 0.0079       & 7.2e02     & 0.0079       \\
            jython         & 1.6e02                             & 0.0079                             & 1.9e02                 & 0.0079                           & 2e02                           & 0.0079                     & 21         & 0.0079       & 1.9e02     & 0.0079       & 7.1e02     & 0.0079       \\
            micropython    & 8.3e02                             & 0.0079                             & 5.3e02                 & 0.0079                           & 5.3e03                         & 0.0079                     & 44         & 0.0079       & 4.3e02     & 0.0079       & 3.6e02     & 0.0079       \\
            nuitka         & 5.4e02                             & 0.0079                             & 4.3e02                 & 0.0079                           & 2e02                           & 0.0079                     & 12         & 0.0079       & 3.6e02     & 0.0079       & 2.2e02     & 0.0079       \\
            numba2         & 35                                 & 0.0079                             & 4e02                   & 0.0079                           & 2.2e02                         & 0.0079                     & 20         & 0.0079       & 14         & 0.0079       & 4.5e02     & 0.0079       \\
            numba3         & 11                                 & 0.0079                             & 4.2e02                 & 0.0079                           & 2.1e02                         & 0.0079                     & 19         & 0.0079       & 10         & 0.0079       & 2.4e02     & 0.0079       \\
            pypy2          & 30                                 & 0.0079                             & 24                     & 0.056                            & \best{1.7e02}                  & \best{best}                & 7.7        & \best{best}  & 27         & 0.0079       & \best{14}  & \best{best}  \\
            pypy3          & 17                                 & 0.0079                             & \best{23}              & \best{best}                      & 2.6e02                         & 0.0079                     & 8.3        & 0.69         & 23         & 0.0079       & 14         & 0.69         \\
            shedskin       & \best{7.8}                         & \best{best}                        &                        &                                  & 3.8e02                         & 0.0079                     & 44         & 0.0079       & 7.1        & \best{best}  & 5.6e02     & 0.0079       \\

            \bottomrule
        \end{tabular}
    }

\end{table*}

