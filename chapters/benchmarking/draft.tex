
\section{Introduction}
Since the objective of this thesis is to study various alternatives for reducing the energy consumption of software programs, this chapter will focus on this topic. We shall conduct extensive empirical research. This chapter describes the instruments required to conduct a successful experiment to measure the energy consumption of software applications.
In addition, the majority of the time we will conduct an analytical comparison; therefore, the goal is to develop a framework to assist practitioners in comparing various solutions in terms of energy consumption.



A reproducible setup environment needs to be established in order to maximize the software's potential to reduce energy consumption.

This chapter will be devoted to giving a set of guidelines with the intention of delivering a test that is accurate, representative, and replicable.

% This chapter will go over each component in further detail, describe it, and explain how it is used in the field of empirical analysis for software energy optimization.




%%%%%%%%% preliminary taughts and section structure 
% The need of reproducibility in our field - software optimization based on empirical studies -

% The importance of Virtualisation for reproducibility \cite{howe_virtual_2012}
% some of the most important parts are
% - fewer constraints on research methods
% - on-demand backups
% - virtual Machines as Publications
% - more Variables captured



%%%% part of the state of the art 
% In the area covered by this PhD thesis, reproducibility might be achieved by ensuring the same execution settings of physical nodes, virtual machines, clusters or cloud environments.
%%% why it won't work for energy consumption 
% However, when it comes to measuring the energy consumption of a system, applying acknowledged guidelines and carefully repeating the same benchmark can nonetheless lead to different energy footprints not only among homogeneous nodes, but even within a single node.
%% reseasons why 
% One major problem that hinders the reproducibility of the empirical benchmarks is the interaction with the external environment, either as concurrency or dependencies.
% Therefore, researchers cannot observe the same results, unless they duplicate the same environment.




%%%% this is relative about my thesis 
In theory, using identical CPU, same memory configuration, similar storage and networking capabilities, should increase the accuracy of physical measurements.
Unfortunately, this is not possible when it comes to measuring the energy consumption of a system.
Applying the benchmarking guidelines and repeating the same experiment with in the same configuration are not sufficient to reproduce the the same energy measurements, not only between identical machines, but even within the same machine.
This difference---also called \emph{energy variation} (EV)---has become a serious threat to the accuracy of experimental evaluations.

Figure~\ref{fig:motivation} illustrates this variation problem as a violin plot of $20$ executions of the benchmark \emph{Conjugate Gradient} (\textsf{CG}) taken from the \emph{NAS Parallel Benchmarks} (NBP) suite~\cite{Bailey:1991:NPB:125826.125925}, on $4$ nodes of an homogeneous cluster (the cluster \textsf{Dahu} described in Table~\ref{table:g5k}) at 50\,\% workload.
We can observe a large variation of the energy consumption, not only among homogeneous nodes, but also at the scale of a single node, reaching up to $25\,\%$ in this example.

\begin{figure}%[!htb]
    \center{\includegraphics[width=.9\linewidth]{imgs/motivation}}
    \caption{CPU energy variation for the benchmark \textsf{CG}}\label{fig:motivation}
\end{figure}
%% state of the art 
Some researchers started investigating the hardware impact of the energy variation of power consumption.
As an example, one can cite~\cite{borkar_designing_2005,tschanz_adaptive_2002} who reported that the main cause of the variation of the power consumption between different machines is due to the \textbf{CMOS} manufacturing process of transistors in a chip.
\cite{heinrich_predicting} described this variation as a set of parameters, such as CPU Frequency and the thermal effect.



As Stephen M. Blackburn~\emph{et~al.} cited in their paper "evaluate collaboratory"~\cite{stephen_evaluate_2012}, one of the major pitfalls of the measurement contexts is the inconsistency, which can be translated here by the fact that the production context is not the same as the benchmarking one.

Another difficult part for practitioners is to generalize the claims they reached beyond the lab conditions.
Are they appropriate?
Are they consistent and are they reproducible?
To answer those questions, the community agreed on some wellknown benchmarks to represent a specific concern of the production world.
One can site as an example the Dacapo~\cite{DaCapo:paper} and Renaissance~\cite{renaissance} benchmark suites for Java applications, or the CLBG benchmark suite for comparing programming languages\footnote{\url{https://benchmarksgame-team.pages.debian.net/benchmarksgame/index.html}}.
Although they do not cover all the cases, the community agrees on their relevance and representativeness.

In addition to these benchmarks, a new category of testing techniques has emerged to simulate the worst cases of the production environments. This new category of benchmarking called performance tests\cite{pradeep2019pragmatic}, which are benchmarks meant to evaluate the behavior of software under stress situations. We can use the Gatling\footnote{\url{https://gatling.io/}} as an illustration for web applications stresser and stress-ng for hardware heavy workload perormance measurements~\cite{king2017stress}.






%%%% THIS Should the introduction of my chapter 

\subsection{Extension}
Morover, The field of computer science is seeing rapid advancements, which has led to an increase in the number of obsolete results. Even more so, when it comes to studies that compare multiple solutions, it is impossible to cover them all.
In addition, between the preliminary experiment and the results that were published, there may have been the appearance of new candidates as well as the development of others. As a result, we would like to provide a fresh idea for a productive experiment that we will call "Extension." The ability to give the required tools to not only repeat the experiment but also to add additional candidates, workloads, or key metrics.


%% how to extend 

We propose making certain enhancements to the benchmarking framework that was suggested by the collaboratory on Experimental Evaluation of Software \footnote{\url{http://evaluate.inf.usi.ch/}}. and Systems in Computer Science in order to address those problems.
Instead of only presenting the four primary aspects of their guidelines, which are \emph{measurement contexts} that indicate the software and hardware components that will alter or remain constant during the experiment. \emph{workloads} which identify the benchmarks to use in the experiment, as well as their inputs; \emph{metrics} that specify the attributes to be measured and how to assess them. \emph{Data analysis} show how to examine data and evaluate the outcomes of the analysis to offer insight into the assertions that arise from the study.

\begin{itemize}
    \item \textbf{candidates}: a set of candidates that will be used in the experiment. The candidates should be agnostic of the experiment context and they all
\end{itemize}





\begin{figure}%[!htb]
    \center{\includegraphics[width=.9\linewidth]{imgs/benchmarkingprotocol}}
    \caption{Benchmarking protocol}\label{fig:benchmarkingprotocol}
\end{figure}



% A more subtle issue may arise due to the values of the measurements that we achieved.
% In fact, the energy measures are quite small, and iterations may be taking a few milliseconds more or less to run.
% A thing we cannot measure using our measurement tools.
% How generalizable are our results? As a set of energy variation optimization guidelines, we argue that our results applied on most of the modern Intel CPU.
% However, using and comparing identical CPU is still tricky and is very dependent to chips.


