% ************************** Thesis Abstract *****************************
% Use `abstract' as an option in the document class to print only the titlepage and the abstract.
\begin{abstract}
    % Wildfires, droughts, heatwaves, floods, melting ice, and other natural calamities have grown alarmingly in recent years, indicating strong evidence of global warming. Climate change is already impacting the entire planet and will only intensify in the future. According to NASA, global temperatures rose by around 1.1 degrees Celsius between 1901 and 2020, mainly due to increased greenhouse gases, which entrap the energy emitted by the sun's rays. While some argue that this happened numerous times in the past due to changes in the Earth's orbit, resulting primarily in five ice ages, this is not the case. Carbon dioxide levels have risen 250 times faster in the last century than in the previous 650,000 years. Furthermore, greenhouse gas concentrations have reached their highest point in 2 million years. 
    % If current trends continue, we will reach +3.2 degrees Celsius by the end of the century, with disastrous repercussions for the Earth. However, according to UN reports, we can still rescue ourselves provided we limit the temperature rise below 1.5 degrees Celsius. As a result, immediate action is required. Many agreements, like the UN Framework Convention on Climate Change and the Paris Agreement, aiming to reach net zero emissions by 2050. They propose three approaches: reducing emissions, adapting to climate impacts, and financing necessary adaptations. 
    The ICT sector is claimed to account for 2\% of global emissions.
    While this may be perceived as a small percentage, the success of ICT technologies inevitably contributes to increase the emissions of greenhouse gases.
    % the ICT sector is the fastest-growing emitter of greenhouse gases, mainly due to the increasing number of data centers. 
    In 2019, the number of data centers worldwide was 1.5 million, which is expected to reach 2.5 million by 2025.
    In this context, reducing data center energy consumption is crucial to lowering the ICT sector's emissions.
    There are three main ways to achieve this goal: improving the hardware's efficiency, reducing the cooling systems' energy consumption, or reducing the energy consumption of the servers themselves.
    This thesis focuses on the last approach, which I believe is the most affordable one, as it does not require any physical changes to data centers.
    My objective is to support developers in producing more sustainable software services through the delivery of tools and guidelines to lower the energy consumption of software running on servers.
    To do so, I decided to follow an empirical approach that consists of three steps: test, measure, then optimize.
    The reason behind such a decision is to keep up with the rapid pace of the software industry.
    Indeed, the software industry is evolving quickly, and keeping up with the latest technologies is challenging.
    Therefore, I decided to provide tools and protocols to help practitioners to test their hypotheses, instead of only reporting them guidelines as I believe that some of the insights shared as part of this thesis might already be obsolete when published.
    Due to the urgency of the matter, I decided to harness the most popular, yet energy-hungry programming language, Python.
    
    In this thesis, I will analyze the energy behavior of Python code during its most commonly used cases.
    Then, I will provide a non-intrusive method to reduce its energy consumption.
    Then, I will extend this strategy to another programming language famous for its legacy code base, Java, to show that we can still reduce the energy consumption of already running applications without paying a considerable price. 
    Finally, I will adopt a more systemic approach.
    Instead of trying to optimize one single application, can one reduce the energy consumption of the data center as a whole entity?
    Thanks to the micro-services architecture, one application can be produced using many services, each independent of the other.
    This releases us from the bond of adopting a single programming language as the monolithic application does.
    And with this, one can use multiple programming languages and take advantage of the strengths of each one for a specific scenario. 
    The last chapter will analyze the energy behavior of several programming languages regarding web services while expanding the principle of timeless applications further in the perspective section.
\end{abstract}
% Recent years had a worrisome increase in the number of wildfires, droughts, heatwaves, floods, melting ice and other natural disatesrs giving us clear sign of global warming. The effects of climate change are already being felt on the whole planet and will continue to worsen in the future. According to NASA, the global temperatures increased by around 1.1 °C from 1901 to 2020. some argues that this cycle has been already happend before, resulting five major ice ages before, and their main cause was the change of distance between earth and sun resulting in an overall alteration in the earth temperature, and the rise of the carbon dioxide level in the atmosphere, this is the first time that we exceed the level of 300 parts / per million was in 1950 within a span of 800000 years, even worse, this number has risen up 420. This level of carbon dioxide is the first within the last 800000. 
% Global warming is the gradual increase of the Earth's average surface temperature. It is caused by the increased levels of greenhouse gases in the atmosphere, which trap energy from the sun and cause the Earth to warm which traps the energy of solar rays in earth leading to an increase on the overall temperature. 
% Carbon dioxide from human activities is increasing about 250 times faster than it did from natural sources after the last Ice Age
% Greenhouse gas concentrations are at their highest levels in 2 million years, and the rate of increase is unprecedented in at least the last 800,000 years.
% it is not too late, according to scientists if we keep the rise of temperature below 1.5 degrees Celsius, we can still avoid the worst effects of climate change. howerver it is projected that the temperature will increase by 3.2 by 2100. 
% According to the Sustainable Development Goals, the UN Framework Convention on Climate Change and the Paris Agreement. Three broad categories of action are: cutting emissions, adapting to climate impacts and financing required adjustments.
% While a growing coalition of countries is committing to net zero emissions by 2050.  